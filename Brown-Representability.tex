\documentclass[12pt]{article}

\usepackage{setspace}

\usepackage{amsmath, amsfonts, amssymb, graphicx, color, fancyhdr, lipsum, scalerel, stackengine, mathrsfs, tikz-cd, mdframed, enumitem, framed, adjustbox, bm, upgreek, xcolor, hyperref}
\usepackage[backend=biber,style=alphabetic]{biblatex}
\usepackage{fullpage}
% this file contains the literature database
\bibliography{sources}
\usepackage[framed,thmmarks]{ntheorem}

% Check out mydefs.tex for my definitions/macros
\input{mydefs.tex}

%header stuff
\setlength{\headsep}{24pt}  % space between header and text
\pagestyle{fancy}     % set pagestyle for document
\lhead{Brown Representability} % put text in header (left side)
\rhead{Courts} % put text in header (right side)
\cfoot{\itshape p. \thepage}
\setlength{\headheight}{15pt}
\allowdisplaybreaks

% Document-Specific Macros
\newcommand*{\dgVectk}{\mathbf{dgVect}_k}
\newcommand*{\dgmodA}{\mathbf{dgmod}\text{-}A}
\newcommand*{\pTopc}{\mathbf{Top}_\ast^c}
\newcommand*{\Set}{\mathbf{Set}}
\newcommand*{\Func}{\mathbf{Func}}
\newcommand*{\pCW}{\mathbf{Top}^{CW}_\ast}
\newcommand*{\relCW}{\mathbf{Top}^{CW}_{\hookrightarrow}}

\begin{document}
%make the title page
\title{Brown Representability\vspace{-1ex}}
\author{Nico Courts}
\date{Math 509 -- Spring 2019}
\maketitle

\renewcommand{\abstractname}{Introduction}
\begin{abstract}
  In this document we write an exposition presenting the topic of Brown representability and how it has appeared in different contexts -- beginning first with some necessary preliminaries followed by a more topological perspective and culminating in a modern and categorical statement holds in triangulated categories.
\end{abstract}
\tableofcontents

\newpage
\section{Outline of Presentation}
The presentation will demonstrate some proper subset of the information in this document. The current sketch is the following:
\begin{itemize}
	\item Introduction and preliminaries
	\begin{itemize}
		\item Background we are assuming (categories, functors, natural transformations, and limits)
		\item What is representability and why is it useful/important?
		\item (Co)homological functors
	\end{itemize}
	\item Brown Representability in Simplicial/CW Complexes
	\begin{itemize}
		\item The homotopy category of pointed connected topological spaces
		\item Categorical notions (e.g. weak pushouts)
		\item Statement of theorem
		\item Application
	\end{itemize}
	\item Brown Representability in Triangulated Categories
	\begin{itemize}
		\item Why are triangulated categories what we want?
		\item Definiton and categorical background (e.g. compact generation)
		\item Statement of theorem
		\item Applications
	\end{itemize}
\end{itemize}

\section{Some Background}
In this section we begin with some preliminary definitions and theorems that serve as the machinery for Brown representability. 
The results that follow are fairly standard and the interested reader can find more information in \cite{neeman-book} and \cite{riehl16}.
For the sake of time, we will omit a discussion of these facts in our lecture but include them in this document in case someone is interested in them.
\subsection{General Category Theory}
\begin{defn}
  A \textbf{category} $\calC$ is a collection of \textbf{objects} along with, for every pair $(X,Y)$ of objects in $\calC$, a collection $\Hom_\calC(X,Y)$ of \textbf{morphisms} from $X$ to $Y$ such that
  \begin{itemize}
    \item There is a distinguished map, the \textbf{identity map} $\id_X\in\Hom_\calC(X,X)$ for each $X\in\calC$ such that, for every $f\in\Hom_\calC(X,Y)$ and every $g\in\Hom_\calC(Y,X)$,
    \[f=\id_X\circ f\qquad g=g\circ\id_X.\]
    \item For each $X,Y,$ and $Z$ in $\calC$ there is a map $\circ:\Hom_\calC(Y,Z)\times\Hom_\calC(X,Y)\to\Hom_\calC(X,Z)$ (called \textbf{composition}) that is associative. That is, for each $f\in\Hom_\calC(X,Y)$, $g\in\Hom_\calC(Y,Z)$, and $h\in\Hom_\calC(Z,W)$,
    \[(h\circ g)\circ f=h\circ(g\circ f).\]
  \end{itemize}
\end{defn}
\begin{defn}
	The \textbf{opposite category} $\calC^{op}$ of a category $\calC$ is the category whose objects are 
	precisely those of $\calC$ and whose morphisms are as follows: for every $A,B\in\calC$ and $f\in\Hom_\calC(A,B)$,
	there is a map $f^\ast\in\Hom_{\calC^{op}}(B,A)$ such that for all composable $f$ and $g$ in $\calC$,
	\[(f\circ g)^\ast=g^\ast\circ f^\ast.\]
\end{defn}

\subsection{Size Considerations}
``Size'' is often an issue when you are discussing category theory in that one can quickly get into trouble with Russell's paradox.
To ameliorate this to some degree, it is often wise to restrict oneself to the categories which are suitably small (although often 
ideas can be extended with extra care to a more general setting).

Here we define some of the types of smallness that we are interested in. The general idea to keep in mind is that if $X$ is small, then the 
collections associated to $X$ are sets.

\begin{defn}
	A category $\calC$ is called \textbf{locally small} if for each $A,B\in\calC$, $\Hom_{\calC}(A,B)$ is a set.
\end{defn}
\begin{defn}
	A category $\calC$ is called \textbf{small} if it is locally small and, in addition, has a set's worth of objects.
\end{defn}
\begin{defn}
	A \textbf{small (co)limit} (see \cite{riehl16} for definition of (co)limits) is a limit of a diagram that is indexed by a small indexing category.
\end{defn}
\begin{rmk}
	Since (co)products are (co)limits, this also gives us a definition for \textbf{small (co)products}. We can also just say 
	they are (co)products indexed by a set.
\end{rmk}

\subsection{Functors and Representability}
\begin{defn}
  A (covariant) \textbf{functor} between two categories $\calC$ and $\calD$ is a map $F:\calC\to\calD$ that maps objects and morphisms in $\calC$ to those in $\calD$, respectively, such that 
  \begin{itemize}
    \item For each $X\in\calC$, we have $F(\id_X)=\id_{F(X)}$
    \item For each $X,Y\in\calC$ and $g\in\Hom_\calC(X,Y)$, $F(g)$ is a morphism in $\Hom_\calC(F(X),F(Y))$.
    \item For all $X,Y,Z$ in $\calC$ and $f\in\Hom_\calC(X,Y)$ and $g\in\Hom_\calC(Y,Z)$,
    \[F(g\circ f)=F(g)\circ F(f)\]
  \end{itemize}
\end{defn}
\begin{rmk}
	Leveraging duality, one can define a \textbf{contravariant functor} $F:\calC\to \calD$ as a covariant functor $F:\calC^{op}\to \calD$.
\end{rmk}
\begin{defn}
	An \textbf{equivalence of categories} $F:\calC\to\calD$ is a functor such that 
	\begin{itemize}
		\item $F$ is both \textbf{full} (injective on $\Hom$s) and \textbf{faithful} (surjective on $\Hom$s); and 
		\item $F$ is \textbf{essentially surjective} -- for each $D\in\calD$, there is a $D'\in\calD$ such that 
		$D\cong D'$ and, for some $C\in\calC$, we have $F(C)=D'$.
	\end{itemize}
	An \textbf{auto-equivalence} for $\calC$ is an equivalence of categories from $\calC$ to itself.
\end{defn}


\subsection{Natural Transformations}
A natural transformation is a ``map between maps'' that demonstrates some kind of nice compatibility between two functors.
Often, when one says a property is ``natural'' what one actually means is that there is a natural transformation involved.

\begin{defn}
	Let $F,G:\calC\to\calD$ be two functors. Then $\eta:F\to G$, defined to be a collection of \textbf{component maps} $\{\eta_\alpha:F(\alpha)\to G(\alpha)\}_{\alpha\in\calC}$, 
	is called a \textbf{natural transformation} if, for every 
	$A,B\in\calC$ and $f\in\Hom_{\calC}(A,B)$, the following square (called a \textbf{naturality square}) commutes:
	\begin{center}
		\begin{tikzcd}
			F(A)\ar[d,"F(f)"]\ar[r,"\eta_A"] & G(A)\ar[d,"G(f)"]\\
			F(B)\ar[r,"\eta_B"] & G(B)
		\end{tikzcd}
	\end{center}
	If $\eta$ has an inverse, then it is called a \textbf{natural isomorphism.}
\end{defn}

\subsection{Triangulated Categories}
The primary context that we are interested is a class of categories called \textit{triangulated categories.} Examples of 
these include $\K(\modR)$, the homotopy category of chain complexes of $R$-modules (modulo chain homotopy) and $\D(\modR)$,
the derived category in which we (Verdier) localize the previous category at quasi-isomorphisms.
\begin{defn}
	Let $\calC$ be a category with auto-equivalence $\Sigma:\calC\to\calC$. A \textbf{triangle} in $\calC$ is a diagram in $\calC$:
	\[X\to Y\to Z\to\Sigma X.\]
\end{defn}
\begin{defn}
	A \textbf{morphism of triangles} is a commutative diagram
	\begin{center}
		\begin{tikzcd}
			X\ar[r,"u"]\ar[d,"f"]& Y\ar[r,"v"]\ar[d,"g"] & Z\ar[r,"w"]\ar[d,"h"] & \Sigma X\ar[d,"\Sigma f"]\\
			X'\ar[r] & Y'\ar[r] & Z'\ar[r] & \Sigma X'
		\end{tikzcd}
	\end{center}
\end{defn}

\begin{lem}
	The collection of all triangles in $\cal C$ with triangle morphisms form a $k$-linear additive category denoted $\Delta(\cal C)$
\end{lem}

\begin{defn}
	The \textbf{rotation} of a triangle $X\xrightarrow{u} Y\to Z\to \Sigma X$ is $Y\to Z\to \Sigma X\xrightarrow{-\Sigma u}\Sigma Y.$
	We use the notation $\mathscr R(\Delta)$ for the rotation of $\Delta$ and $\scrR^{-1}$ for the inverse rotation.
\end{defn}

\begin{defn}
	Let $\cal C$ be a $k$-linear additive category with a \textbf{suspension} or \textbf{shift} functor (auto-equivalence, actually)
	\[\Sigma:\cal C\to\cal C.\]

	Then $({\calC}, \Sigma ,D)$ is \textbf{pre-triangulated} category where $D$ is a full, nonempty subcategory $D\subseteq\Delta(\cal C)$
	with shift functor $\Sigma$ of $D$ and we have the following axioms:
	\begin{itemize}
		\item (\textbf{TR0}) $0\to X\xrightarrow{\id} X\to 0$ is in $D$ for each $X\in \cal C$ and furthermore $D$ is closed under both shifts and triangle isomorphisms.
		\item (\textbf{TR1}) [\textit{Mapping Cone Axiom}] For any $f:X\to Y$ in $\cal C$, there is a triangle
		\[X\xrightarrow{f} Y\to Z\to\Sigma X\]
		\item (\textbf{TR2}) [\textit{Rotation Axiom}] If $F\in D$, then $\mathscr R(F),\mathscr R^{-1}(F)\in D$
		\item (\textbf{TR3}) [\textit{Morphism Axiom}] Given two triangles 
		\begin{center}
			\begin{tikzcd}
				X\ar[r]\ar[d,"f"] & Y\ar[r]\ar[d,"g"] & Z\ar[r]\ar[d,"h",dashed] & \Sigma X\ar[d,"\Sigma f"]\\
				X'\ar[r] & Y'\ar[r] & Z'\ar[r] & \Sigma X'
			\end{tikzcd}
		\end{center}
		with maps $f$ and $g$, there exists an $h:Z\to Z'$ such that the above diagram commutes.
	\end{itemize}
\end{defn}
\begin{rmk}
	The name ``mapping cone axiom'' in (\textbf{TR2}) above comes from the fact that in $\K(\modR)$, or in fact in 
	any additive category with suspension, $Z$ can be constructed as $\cone(f)$, the mapping cone of $f$. 
\end{rmk}
\begin{defn}
	A triple $(\calC,\Sigma, D)$ is a \textbf{triangulated category} if it is pre-triangulated and, in addition,
	satisfies the axiom (\textbf{TR4}) (Verdier/octahedral axiom):

	Suppose we are given three triangles: $X\xrightarrow{u}Y\to Z'\to\Sigma X$, $Y\xrightarrow{v}Z\to X'\to\Sigma Y$
	and $X\xrightarrow{v\circ u} Z\to Y'\to\Sigma X$. Then there is a triangle $Z'\to Y'\to X'\to\Sigma Z'$ such that
	\begin{center}
		\begin{tikzcd}
			X\ar[dr,"u"]\ar[rr,"v\circ u",bend left] && Z\ar[rr,bend left]\ar[dr] && X'\ar[rr,dashed,bend left]\ar[dr] && \Sigma Z'\\
			& Y\ar[ur,"v"]\ar[dr] && Y'\ar[ur,dashed]\ar[dr] && \Sigma Y\ar[ur] &\\
			&& Z'\ar[rr,bend right]\ar[ur,dashed] && \Sigma X\ar[ur] &&
		\end{tikzcd}
	\end{center}
\end{defn}
\begin{rmk}
There are two other representations of the octahedral axiom that are sometimes more helpful:
\begin{center}
	\begin{tikzcd}
		X\ar[r,"u"]\ar[d,"\sim"] & Y\ar[r]\ar[d,"v"] & Z'\ar[r]\ar[d,dashed] &\Sigma X\ar[d,"\sim"]\\
		X\ar[r,"v\circ u"] & Z\ar[r]\ar[d] & Y'\ar[r]\ar[d,dashed] &\Sigma X\ar[d,"\Sigma u"]\\
		& X'\ar[r,"\sim"] \ar[d] & X'\ar[r]\ar[d,dashed] & \Sigma Y\\
		&\Sigma Y\ar[r] & \Sigma Z' &
	\end{tikzcd}
\end{center}

\begin{center}
	\begin{tikzcd}
		&& & X\ar[dl,"u"]\ar[dd,"v\circ u"] &\\
		&& Y\ar[dr,"v"]\ar[ddll] & &\\
		&& & Z\ar[d]\ar[dr] &\\
		Z'\ar[rrr,dashed] && & Y'\ar[r,dashed] & X'
	\end{tikzcd}
\end{center}
\end{rmk}

We often want that our triangulated category $\calT$ is slightly more controlled. For this, we introduce another 
condition:
\begin{defn}
  If $\calT$ is a triangulated category that is closed under small coproducts, then it is said to satisfy (\textbf{TR5}). If $\calT^{op}$ satisfies (\textbf{TR5}) (or equivalently $\calT$ is closed under small products), then we say $\calT$ satisfies (\textbf{TR5}*).
\end{defn}

\section{Brown Representability for Simpilicial Complexes}
One of the first incarnations of Brown representability comes from topology (as these things often do). First we need to single out 
the appropriate context.

\subsection{Category of Interest}
\begin{defn}
	Let $\pTopc$ denote the category of connected pointed topological spaces. That is, the objects are pairs $(X,p)$ where $X$ is a connected 
	topological space and $p$ is a distinguished point in $X$. The maps are continuous maps that send the distinguished point of one pair to the distinguished point in the other.
\end{defn}
\begin{defn}
	$\text{Ho}(\pTopc)$ is the \textbf{homotopy category of connected topological spaces.} The objects are the same as in $\pTopc$, 
	but maps are \textit{weak homotopy equivalence classes} (relative to the basepoint) of maps of pointed spaces.
\end{defn}
\begin{rmk}
	In essence, $\text{Ho}(\pTopc)$ consists of weak homotopy equivalence classes (recall that two objects are weak homotopy equivalent if they have the same 
	homotopy groups).
\end{rmk}
\begin{rmk}
	This is a form of \textit{localization of categories} that is outside the scope of this paper, but closely mirrors the idea 
	we see in commutative algebra.
\end{rmk}
\begin{rmk}
	Since any topological space is weakly homotopy equivalent to a CW-complex, we can easily see that 
	$\text{Ho}(\pTopc)\simeq\text{Ho}(\mathbf{CW}_\ast)$, the homotopy category of pointed CW complexes. This gives us some nice leverage for computation and tangibility.
\end{rmk}

\subsection{Generalized (Eilenberg-Steenrod) Cohomology Theories}
The idea here is that we can construct a general definition that generally captures what we mean when we say we are 
working with cohomology.

Notice that these theories notably do \textit{not} include nonabelian cohomology theories and other theories of interest.
\begin{defn}
	$\Ab^\bbZ$ is the functor category $\Func(\bbZ,\Ab)$ or, equivalently, the category of $\bbZ$-graded Abelian groups.
\end{defn}
\begin{rmk}
	In case you haven't seen it before, the category $\bbZ$ we're talking about has as its objects the integers and 
	a (unique) morphism $m\to n$ whenever $n\ge m$. This is the category we can construct from any set that admits a preorder (reflexivity and transitivity).

	Functorality here says that for any $k\in\bbN$ we can consider the degree $k$ maps between the $i^{th}$ graded piece 
	of $G$ and the $(i+k)^{th}$ graded piece of $H$ for any $G,H\in\Ab^{\bbZ}$. If $\varphi:G\to H$ is any degree zero map and $h\in H^k$ is 
	any degree $k$ element in $H$, then an example of such a map is $\varphi'(g)=h\cdot \varphi(g).$
\end{rmk}
\begin{defn}[(Reduced) Generalized Cohomology Theory]\label{def-red-cohom}
	A \textbf{reduced generalized cohomology theory} on $\calC=\pCW$ is 
	a family of functors:
	\[\tilde E^\bullet:\text{Ho}(\calC)^{op}\to\Ab^\bbZ\]
	along with a collection of natural isomorphisms
	\[\eta^i:\tilde E^i\to \tilde E^{i+1}\circ\Sigma\]
	subject to the following conditions:
	\begin{itemize}
		\item (\textit{Homotopy Invariance})If $f,g:X\to Y$ are homotopic maps (relative to their base points), then \[\tilde E^\bullet(f)=\tilde E^\bullet(g)\]
		\item (\textit{Exactness}) For the short exact sequence $A\stackrel{\iota}{\hookrightarrow} X\to \cone(\iota)$ the following is exact:
		\[\tilde E^\bullet(\cone(\iota))\to \tilde E^\bullet(X)\xrightarrow{\iota^\ast}\tilde E^\bullet(A)\]
	\end{itemize}
\end{defn}

Reduced cohomology theories more succinctly capture the notions we expect when we are doing cohomology,
but in order to do things like relative homology, one needs a little more generality.

Let $\relCW$ be the category of pairs $(X,A)$ of CW complexes with inclusions $A\hookrightarrow X$ with subspace-preserving 
morphisms.

\begin{defn}[(Unreduced, Relative) Generalized Cohomology Theory]
	An unreduced, relative generalized cohomology theory for $\relCW$ is a collection of functors 
	\[E^\bullet:\text{Ho}(\relCW)\to\Ab^\bbZ\]
	along with natural isomorphisms
	\[\eta_{(X,A)}:E^\bullet(A,\varnothing)\to E^{\bullet+1}(X,A)\]
	such that it satisfies an analogous condition for homotopy invariance that we saw in \ref{def-red-cohom} along with 
	the following: 
	\begin{itemize}
		\item (\textit{Exactness}) For $(X,A)$, the following is exact:
		\[\cdots\to E^n(X,A)\to E^n(X,\varnothing)\to E^n(A,\varnothing)\xrightarrow{\eta_{(X,A)}}E^{n+1}(X,A)\to\cdots\]
		or, more succinctly,
		\[\cdots\to E^n(X,A)\to E^n(X)\to E^n(A)\to E^{n+1}(X,A)\]
		which reflects the long exact sequence in relative cohomology that you may be familiar with (see your favorite book on algebraic topology, e.g.
		\cite[p.199]{hatcher}).
		\item (\textit{Excision}) For $(X,A)$ and $(A,U)$ in $\relCW$ such that $\overline{U}\subseteq \text{Int}(A),$ the
		inclusion $\iota:(X\setminus U,A\setminus U)\hookrightarrow (X,A)$ induces an isomorphism
		\[\iota^\ast:E^\bullet(X,A)\xrightarrow{\simeq}E^\bullet(X\setminus U,A\setminus U)\]
	\end{itemize}
\end{defn}

\subsection{Stable Homotopy Categories}
Here we will throw down a few definitions and ideas that illustrate the idea that \textit{Eilenberg-Steenrod cohomology theories are 
natural constructions} in that they are precisely the cohomology of the $(\infty,1)$ category of spectra.

\subsection{Ancillary Definitions}
\begin{defn}
	A \textbf{weak pushout} of a diagram $X\xleftarrow{f}Y\xrightarrow{g}Z$ is an object $W$ and maps 
	$f':X\to W$ and $g':Z\to W$ such that for any object $O$ and maps $\alpha:X\to O$ and $\beta:Z\to O$, 
	such that $\alpha\circ f=\beta\circ g$, there exists a map $\gamma:W\to O$ such that the following diagram commutes:
	\begin{center}
		\begin{tikzcd}
			Y\ar[r,"f"]\ar[d,"g"] & X\ar[d,"f'"]\ar[ddr,"\alpha",bend left] &\\
			Z\ar[r,"g'"]\ar[rrd,"\beta",bend right] & W\ar[dr,dashed,"\gamma"] & \\
			& & O 
		\end{tikzcd}
	\end{center}

	A \textbf{weak pullback} is defined dually.
\end{defn}
\begin{rmk}
	The ``weakness'' referred to in this definition is due to the fact that the map $\gamma$ exists, but is not 
	required to be unique (unlike the regular pullback and pushout).
\end{rmk}

\subsection{The Theorem}
The following result was proven by Brown and presented in \cite{brown65} and further generalized in \cite{adams71}.
\begin{thm}[Brown-Adams]
	A contravariant functor $F:\text{Ho}(\pTopc)\to\mathbf{Set}_\ast$ is representable precisely if 
	\begin{itemize}
		\item $F$ takes coproducts to products; and 
		\item $F$ takes weak pushouts to weak pullbacks.
	\end{itemize}
\end{thm}

\subsection{What This Means}
What should we take away from Brown representability? If we accept the hypothesis that generalized cohomology 
theories are reasonable approximations of what we are looking for in cohomology, this implies that each such 
incarnation of a cohomology theory $H^\bullet$ admits a spectrum object $E_\bullet$ in the stable homotopy category 
such that $H^\bullet\cong \text{Ho}(\pCW)(-,E_\bullet)$ such that the suspension isomorphism 
$\eta:H^n(-)\to H^{n+1}(\Sigma(-))$ is induced from the structure morphisms $\sigma_n:E_n\to\Omega E_{n+1}$ of the spectrum:
\[H^n(-)\cong \text{Ho}(\pCW)(-,E_n)\xrightarrow{(\sigma_n)}\text{Ho}(\pCW)(-,\Omega E_{n+1})\cong\text{Ho}(\pCW)(\Sigma(-),E_{n+1})\cong H^{n+1}(\sigma(-))\]

\section{Brown Representability for Triangulated Categories}
\subsection{Additional Notions}
Along with the standard categorical definitions we saw in section one, we need one further restriction on our categories as formulated in \cite{neeman-article}:
\begin{defn}[Compact Object]
	Let $X\in\calC$ be an object in a triangulated category. Then $X$ is called \textbf{compact} if for every coproduct of objects in $\calC$
	\[\Hom_\calC\left(X,\coprod_{\lambda\in\Lambda}t_\lambda\right)=\coprod_{\lambda\in\Lambda}\Hom_\calC(X,t_\lambda)\]
\end{defn}
\begin{rmk}
	The idea here is that when the codomain is a product, this equality always exists. The projections give you 
	a product of maps into the factors and the universal property of products gives you a (unique!) map into the product if you
	have maps into each of the factors.

	This does not hold in general for coproducts, however! Recall that in additive categories 
	one has a canonical isomorphism between finite products and coproducts. Since triangulated 
	categories are additive, this is akin to saying that $X$ is well-behaved in that the maps 
	from $X$ still split even when the codomain is an arbitrary coproduct.
\end{rmk}
\begin{defn}[Compactly Generated Triangulated Category]
	Let $\calC$ be a triangulated category. Then $\calC$ is called \textbf{compactly generated by a set $G$} of its elements 
	if $\calC$ is closed under small coproducts and $G$ consists of compact objects of $\calC$ such that for all $X\in\calC$,
	\[\Hom_\calC(G,X)=0\quad\Rightarrow\quad X=0.\]
\end{defn}
\begin{rmk}
	Here the generators of $\calC$ are elements that have ``sufficient complexity'' to detect all nonzero elements. 
	
	In the more general context where $\calC$ is not necessarily additive, we can generalize this definition to a set of objects 
	such that for every generator $E\in G$ and any $X,Y\in\calC$, if for every $f,g\in\Hom(X,Y)$ and $h\in\Hom(E,X)$ we have 
	\[f\circ e=g\circ e\]
	then 
	\[f=g\]
	and here we would say that $G$ has ``sufficient complexity'' to differentiate between any two non-identical maps.
\end{rmk}
\begin{ex}
	In $\Rmod$, a set of generators is the collection of all free $R$-modules or, equivalently, the collection 
	\[\{R^S|S\in\Set\setminus\varnothing\}\]
	where if $S\subset M\in\Rmod$ is any nonempty subset, then $R^S$ is the product of $|S|$ copies of $R$, 
	where a nonzero map to $M$ is the map which sends the identity in factor to a different $s\in S$.
\end{ex}
\begin{defn}\label{def-rep-func}
	Let $F:\calC\to \mathbf{Set}$ be a (covariant) functor. Then $F$ is called \textbf{representable} if there exists an object $A\in \calC$ such that
	$F$ is naturally isomorphic to $\Hom_{\calC}(-,A)$.
\end{defn}
\begin{defn}\label{def-cohom-func}
	Let $R$ be a ring and let $\calT$ be a triangulated category. Then a functor $F:\calT\to\Rmod$ is called \textbf{cohomological}
	if $F$ sends each triangle $A\to B\to C\to\Sigma A$ to an exact sequence in $\Rmod$.
\end{defn}
\begin{rmk}
	This definition extends to functors from triangulated categories to any Abelian category, but due to the Freyd-Mitchell
	embedding it suffices to use $\Rmod$ for all small examples.
\end{rmk}


\subsection{The Statement}
Finally we come to the full-powered version of the Brown Representability due to Neeman in \cite{neeman-article}:
\begin{thm}[Brown]
	Let $\calT$ be a compactly generated triangulated category. For a functor $H:\calT^{op}\to\Ab$, the following 
	are equivalent:
	\begin{itemize}
		\item The functor $H$ is cohomological (def~\ref{def-cohom-func}) and preserves small coproducts.
		\item $H$ is representable (def~\ref{def-rep-func}).
	\end{itemize}
\end{thm}

%%%%%%%%%%%%%%%%%%%%%%%%%%%%%%%%%%%%%%
%%%%%%%%%%  Bibliography %%%%%%%%%%%%%
%%%%%%%%%%%%%%%%%%%%%%%%%%%%%%%%%%%%%%
\medskip

\printbibliography

\end{document}
