\documentclass[12pt]{article}

\usepackage{setspace}

\usepackage{amsmath, amsfonts, amssymb, graphicx, color, fancyhdr, lipsum, scalerel, stackengine, mathrsfs, tikz-cd, mdframed, enumitem, framed, adjustbox, bm, upgreek, xcolor, hyperref}
\usepackage[backend=biber,style=alphabetic]{biblatex}
\usepackage{fullpage}
% this file contains the literature database
\bibliography{sources}
\usepackage[framed,thmmarks]{ntheorem}

% Check out mydefs.tex for my definitions/macros
\input{mydefs.tex}

%header stuff
\setlength{\headsep}{24pt}  % space between header and text
\pagestyle{fancy}     % set pagestyle for document
\lhead{Brown Representability} % put text in header (left side)
\rhead{Cheng \& Courts} % put text in header (right side)
\cfoot{\itshape p. \thepage}
\setlength{\headheight}{15pt}
\allowdisplaybreaks

% Document-Specific Macros
\newcommand*{\dgVectk}{\mathbf{dgVect}_k}
\newcommand*{\dgmodA}{\mathbf{dgmod}\text{-}A}

\begin{document}
%make the title page
\title{Brown Representability\vspace{-1ex}}
\author{Everett Cheng and Nico Courts}
\date{Math 509 -- Spring 2019}
\maketitle

\renewcommand{\abstractname}{Introduction}
\begin{abstract}
  In this document we write an exposition presenting the topic of Brown representability and how it has appeared in different contexts -- beginning first with some necessary preliminaries followed by a more topological perspective and culminating in a modern and categorical statement holds in triangulated categories.
\end{abstract}

\section{Some Background and Motivation}
In this section we begin with some preliminary definitions and theorems that serve as the machinery for Brown representability. The results that follow are fairly standard and the interested reader can find more information in \cite{neeman-book} and \cite{riehl}.
\subsection{General Category Theory}
\begin{defn}
  A \textbf{category} $\calC$ is a collection of \textbf{objects} along with, for every pair $(X,Y)$ of objects in $\calC$, a collection $\Hom_\calC(X,Y)$ of \textbf{morphisms} from $X$ to $Y$ such that
  \begin{itemize}
    \item There is a distinguished map, the \textbf{identity map} $\id_X\in\Hom_\calC(X,X)$ for each $X\in\calC$ such that, for every $f\in\Hom_\calC(X,Y)$ and every $g\in\Hom_\calC(Y,X)$,
    \[f=\id_X\circ f\qquad g=g\circ\id_X.\]
    \item For each $X,Y,$ and $Z$ in $\calC$ there is a map $\circ:\Hom_\calC(Y,Z)\times\Hom_\calC(X,Y)\to\Hom_\calC(X,Z)$ (called \textbf{composition}) that is associative. That is, for each $f\in\Hom_\calC(X,Y)$, $g\in\Hom_\calC(Y,Z)$, and $h\in\Hom_\calC(Z,W)$,
    \[(h\circ g)\circ f=h\circ(g\circ f).\]
  \end{itemize}
\end{defn}
\begin{defn}
  A (covariant) \textbf{functor} between two categories $\calC$ and $\calD$ is a map $F:\calC\to\calD$ that maps objects and morphisms in $\calC$ to those in $\calD$, respectively, such that 
  \begin{itemize}
    \item For each $X\in\calC$, we have $F(\id_X)=\id_{F(X)}$
    \item For each $X,Y\in\calC$ and $g\in\Hom_\calC(X,Y)$, $F(g)$ is a morphism in $\Hom_\calC(F(X),F(Y))$.
    \item For all $X,Y,Z$ in $\calC$ and $f\in\Hom_\calC(X,Y)$ and $g\in\Hom_\calC(Y,Z)$,
    \[F(g\circ f)=F(g)\circ F(f)\]
  \end{itemize}

\end{defn}


\subsection{Triangulated Categories}
The primary context that we are interested is a class of categories called \textit{triangulated categores.} Examples of 
these include $H^0(\Ch(\modR))$, the homotopy category of chain complexes of $R$-modules (modulo chain homotopy) and $\D(\modR)$,
the derived category in which we (Verdier) localize the previous category at quasi-isomorphisms.
\begin{defn}
	Let $\calC$ be a category with auto-equivalence $\Sigma:\calC\to\calC$. A \textbf{triangle} in $\calC$ is a diagram in $\calC$:
	\[X\to Y\to Z\to\Sigma X.\]
\end{defn}
\begin{defn}
	A \textbf{morphism of triangles} is a commutative diagram
	\begin{center}
		\begin{tikzcd}
			X\ar[r,"u"]\ar[d,"f"]& Y\ar[r,"v"]\ar[d,"g"] & Z\ar[r,"w"]\ar[d,"h"] & \Sigma X\ar[d,"\Sigma f"]\\
			X'\ar[r] & Y'\ar[r] & Z'\ar[r] & \Sigma X'
		\end{tikzcd}
	\end{center}
\end{defn}

\begin{lem}
	The collection of all triangles in $\cal C$ with triangle morphisms form a $k$-linear additive category denoted $\Delta(\cal C)$
\end{lem}

\begin{defn}
	The \textbf{rotation} of a triangle $X\xrightarrow{u} Y\to Z\to \Sigma X$ is $Y\to Z\to \Sigma X\xrightarrow{-\Sigma u}\Sigma Y.$
	We use the notation $\mathscr R(\Delta)$ for the rotation of $\Delta$.
\end{defn}

\begin{defn}
	Let $\cal C$ be a $k$-linear additive category with a \textbf{suspension} or \textbf{shift} functor (auto-equivalence, actually)
	\[\Sigma:\cal C\to\cal C.\]

	Then $({\calC}, \Sigma ,D)$ is \textbf{pre-triangulated} category where $D$ is a full, nonempty subcategory $D\subseteq\Delta(\cal C)$
	with shift functor $\Sigma$ of $D$ and we have the following axioms:
	\begin{itemize}
		\item (\textbf{TR0}) $X\xrightarrow{\id} X\to 0$ is in $D$ for each $X\in \cal C$ and furthermore $D$ is closed under both shifts and triangle isomorphisms.
		\item (\textbf{TR1}) [\textit{Mapping Cone Axiom}] For any $f:X\to Y$ in $\cal C$, there is a triangle
		\[X\xrightarrow{f} Y\to Z\to\Sigma X\]
		\item (\textbf{TR2}) [\textit{Rotation Axiom}] If $F\in D$, then $\mathscr R(F),\mathscr R^{-1}(F)\in D$
		\item (\textbf{TR3}) [\textit{Morphism Axiom}] Given two triangles 
		\begin{center}
			\begin{tikzcd}
				X\ar[r]\ar[d,"f"] & Y\ar[r]\ar[d,"g"] & Z\ar[r]\ar[d,"h",dashed] & \Sigma X\ar[d,"\Sigma f"]\\
				X'\ar[r] & Y'\ar[r] & Z'\ar[r] & \Sigma X'
			\end{tikzcd}
		\end{center}
		with maps $f$ and $g$, there exists an $h:Z\to Z'$ such that the above diagram commutes.
	\end{itemize}
\end{defn}
\begin{defn}
	A triple $(\calC,\Sigma, D)$ is a \textbf{triangulated category} if it is pre-triangulated and, in addition,
	satisfies the axiom (\textbf{TR4}) (Verdier/octahedral axiom):

	Suppose we are given three triangles: $X\xrightarrow{u}Y\to Z'\to\Sigma X$, $Y\xrightarrow{v}Z\to X'\to\Sigma Y$
	and $X\xrightarrow{v\circ u} Z\to Y'\to\Sigma X$. Then there is a triangle $Z'\to Y'\to X'\to\Sigma Z'$ such that
	\begin{center}
		\begin{tikzcd}
			X\ar[dr,"u"]\ar[rr,"v\circ u",bend left] && Z\ar[rr,bend left]\ar[dr] && X'\ar[rr,dashed,bend left]\ar[dr] && \Sigma Z'\\
			& Y\ar[ur,"v"]\ar[dr] && Y'\ar[ur,dashed]\ar[dr] && \Sigma Y\ar[ur] &\\
			&& Z'\ar[rr,bend right]\ar[ur,dashed] && \Sigma X\ar[ur] &&
		\end{tikzcd}
	\end{center}
\end{defn}
\begin{rmk}
There are two other representations of the octahedral axiom that are sometimes more helpful:
\begin{center}
	\begin{tikzcd}
		X\ar[r,"u"]\ar[d,"\sim"] & Y\ar[r]\ar[d,"v"] & Z'\ar[r]\ar[d,dashed] &\Sigma X\ar[d,"\sim"]\\
		X\ar[r,"v\circ u"] & Z\ar[r]\ar[d] & Y'\ar[r]\ar[d,dashed] &\Sigma X\ar[d,"\Sigma u"]\\
		& X'\ar[r,"\sim"] \ar[d] & X'\ar[r]\ar[d,dashed] & \Sigma Y\\
		&\Sigma Y\ar[r] & \Sigma Z' &
	\end{tikzcd}
\end{center}

\begin{center}
	\begin{tikzcd}
		&& & X\ar[dl,"u"]\ar[dd,"v\circ u"] &\\
		&& Y\ar[dr,"v"]\ar[ddll] & &\\
		&& & Z\ar[d]\ar[dr] &\\
		Z'\ar[rrr,dashed] && & Y'\ar[r,dashed] & X'
	\end{tikzcd}
\end{center}
\end{rmk}
\begin{defn}
  If $\calT$ is a triangulated category that is closed under small coproducts, then it is said to satisfy (\textbf{TR5}). If $\calT^{op}$ satisfies (\textbf{TR5}) (or equivalently $\calT$ is closed under small products), then we say $\calT$ satisfies (\textbf{TR5}*).
\end{defn}




\section{Brown Representability for Simpilicial Complexes}

\section{Brown Representability for Triangulated Categories}

%%%%%%%%%%%%%%%%%%%%%%%%%%%%%%%%%%%%%%
%%%%%%%%%%  Bibliography %%%%%%%%%%%%%
%%%%%%%%%%%%%%%%%%%%%%%%%%%%%%%%%%%%%%
\medskip

\printbibliography

\end{document}
