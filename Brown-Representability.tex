\documentclass[12pt]{article}

\usepackage{setspace}

\usepackage{amsmath, amsfonts, amssymb, graphicx, color, fancyhdr, lipsum, scalerel, stackengine, mathrsfs, tikz-cd, mdframed, enumitem, framed, adjustbox, bm, upgreek, xcolor, hyperref}
\usepackage[backend=biber,style=alphabetic]{biblatex}
\usepackage{fullpage}
% this file contains the literature database
\bibliography{sources}
\usepackage[framed,thmmarks]{ntheorem}

% Check out mydefs.tex for my definitions/macros
%set up theorem/definition/etc envs
%Problems will be created using their own counter and style
\theoremstyle{break}
\theoreminframepreskip{0pt}
\theoreminframepostskip{0pt}
\newframedtheorem{prob}{Problem}[section]

%solution template
\theoremstyle{nonumberbreak}
\theoremindent0.5cm
\theorembodyfont{\upshape}
\theoremseparator{:}
\theoremsymbol{\ensuremath\spadesuit}
\newtheorem{sol}{Solution}

%Theorems
\definecolor{thmcol}{RGB}{120,100,50}
\theoremstyle{changebreak}
\theoremseparator{}
\theoremsymbol{}
\theoremindent0.5cm
\theoremheaderfont{\color{thmcol}\bfseries} 
\newtheorem{thm}{Theorem}[subsection]

%Lemmas and Corollaries
\theoremheaderfont{\bfseries}
\newtheorem{lem}[thm]{Lemma}
\newtheorem{cor}[thm]{Corollary}
\newtheorem{prop}[thm]{Proposition}

%Create a new env that references a theorem and creates a 'primed' version
%Note this can be used recursively to get double, triple, etc primes
\newenvironment{thm-prime}[1]
  {\renewcommand{\thethm}{\ref{#1}$'$}%
   \addtocounter{thm}{-1}%
   \begin{thm}}
  {\end{thm}}

\setlength\fboxsep{15pt}

%Shade definitions
\theoremindent0cm
\theoremheaderfont{\normalfont\bfseries} 
\def\theoremframecommand{\colorbox[rgb]{1,0.9,.55}}
\newshadedtheorem{defn}[thm]{Definition}

%Example
\theoremstyle{break}
\def\theoremframecommand{\colorbox[rgb]{0.9,0.9,0.9}}
\newshadedtheorem{ex}{Example}[section]

%Man, that's really good! Let's use the same thing for definitons.
\newenvironment{def-prime}[1]
  {\renewcommand{\thethm}{\ref{#1}$'$}%
   \addtocounter{thm}{-1}%
   \begin{def}}
  {\end{def}}

%proofs
\theoremstyle{nonumberbreak}
\theoremindent0.5cm
\theoremheaderfont{\sc}
\theoremseparator{}
\theoremsymbol{\ensuremath\spadesuit}
\newtheorem{prf}{Proof}

\theoremstyle{nonumberplain}
\theoremseparator{:}
\theoremsymbol{}
\newtheorem{conj}{Conjecture}

%remarks
\theoremstyle{change}
\theoremindent0.5cm
\theoremheaderfont{\sc}
\theoremseparator{:}
\theoremsymbol{}
\newtheorem{rmk}[thm]{Remark}

%Put page breaks before each part
\let\oldpart\part%
\renewcommand{\part}{\clearpage\oldpart}%

% Blackboard letters
\newcommand*{\bbA}{\mathbb{A}}
\newcommand*{\bbB}{\mathbb{B}}
\newcommand*{\bbC}{\mathbb{C}}
\newcommand*{\bbD}{\mathbb{D}}
\newcommand*{\bbE}{\mathbb{E}}
\newcommand*{\bbF}{\mathbb{F}}
\newcommand*{\bbG}{\mathbb{G}}
\newcommand*{\bbH}{\mathbb{H}}
\newcommand*{\bbI}{\mathbb{I}}
\newcommand*{\bbJ}{\mathbb{J}}
\newcommand*{\bbK}{\mathbb{K}}
\newcommand*{\bbL}{\mathbb{L}}
\newcommand*{\bbM}{\mathbb{M}}
\newcommand*{\bbN}{\mathbb{N}}
\newcommand*{\bbO}{\mathbb{O}}
\newcommand*{\bbP}{\mathbb{P}}
\newcommand*{\bbQ}{\mathbb{Q}}
\newcommand*{\bbR}{\mathbb{R}}
\newcommand*{\bbS}{\mathbb{S}}
\newcommand*{\bbT}{\mathbb{T}}
\newcommand*{\bbU}{\mathbb{U}}
\newcommand*{\bbV}{\mathbb{V}}
\newcommand*{\bbW}{\mathbb{W}}
\newcommand*{\bbX}{\mathbb{X}}
\newcommand*{\bbY}{\mathbb{Y}}
\newcommand*{\bbZ}{\mathbb{Z}}
%Fraktur letters
\newcommand*{\frakA}{\mathfrak{A}}
\newcommand*{\frakB}{\mathfrak{B}}
\newcommand*{\frakC}{\mathfrak{C}}
\newcommand*{\frakD}{\mathfrak{D}}
\newcommand*{\frakE}{\mathfrak{E}}
\newcommand*{\frakF}{\mathfrak{F}}
\newcommand*{\frakG}{\mathfrak{G}}
\newcommand*{\frakH}{\mathfrak{H}}
\newcommand*{\frakI}{\mathfrak{I}}
\newcommand*{\frakJ}{\mathfrak{J}}
\newcommand*{\frakK}{\mathfrak{K}}
\newcommand*{\frakL}{\mathfrak{L}}
\newcommand*{\frakM}{\mathfrak{M}}
\newcommand*{\frakN}{\mathfrak{N}}
\newcommand*{\frakO}{\mathfrak{O}}
\newcommand*{\frakP}{\mathfrak{P}}
\newcommand*{\frakQ}{\mathfrak{Q}}
\newcommand*{\frakR}{\mathfrak{R}}
\newcommand*{\frakS}{\mathfrak{S}}
\newcommand*{\frakT}{\mathfrak{T}}
\newcommand*{\frakU}{\mathfrak{U}}
\newcommand*{\frakV}{\mathfrak{V}}
\newcommand*{\frakW}{\mathfrak{W}}
\newcommand*{\frakX}{\mathfrak{X}}
\newcommand*{\frakY}{\mathfrak{Y}}
\newcommand*{\frakZ}{\mathfrak{Z}}
\newcommand*{\fraka}{\mathfrak{a}}
\newcommand*{\frakb}{\mathfrak{b}}
\newcommand*{\frakc}{\mathfrak{c}}
\newcommand*{\frakd}{\mathfrak{d}}
\newcommand*{\frake}{\mathfrak{e}}
\newcommand*{\frakf}{\mathfrak{f}}
\newcommand*{\frakg}{\mathfrak{g}}
\newcommand*{\frakh}{\mathfrak{h}}
\newcommand*{\fraki}{\mathfrak{i}}
\newcommand*{\frakj}{\mathfrak{j}}
\newcommand*{\frakk}{\mathfrak{k}}
\newcommand*{\frakl}{\mathfrak{l}}
\newcommand*{\frakm}{\mathfrak{m}}
\newcommand*{\frakn}{\mathfrak{n}}
\newcommand*{\frako}{\mathfrak{o}}
\newcommand*{\frakp}{\mathfrak{p}}
\newcommand*{\frakq}{\mathfrak{q}}
\newcommand*{\frakr}{\mathfrak{r}}
\newcommand*{\fraks}{\mathfrak{s}}
\newcommand*{\frakt}{\mathfrak{t}}
\newcommand*{\fraku}{\mathfrak{u}}
\newcommand*{\frakv}{\mathfrak{v}}
\newcommand*{\frakw}{\mathfrak{w}}
\newcommand*{\frakx}{\mathfrak{x}}
\newcommand*{\fraky}{\mathfrak{y}}
\newcommand*{\frakz}{\mathfrak{z}}
% Caligraphic letters
\newcommand*{\calA}{\mathcal{A}}
\newcommand*{\calB}{\mathcal{B}}
\newcommand*{\calC}{\mathcal{C}}
\newcommand*{\calD}{\mathcal{D}}
\newcommand*{\calE}{\mathcal{E}}
\newcommand*{\calF}{\mathcal{F}}
\newcommand*{\calG}{\mathcal{G}}
\newcommand*{\calH}{\mathcal{H}}
\newcommand*{\calI}{\mathcal{I}}
\newcommand*{\calJ}{\mathcal{J}}
\newcommand*{\calK}{\mathcal{K}}
\newcommand*{\calL}{\mathcal{L}}
\newcommand*{\calM}{\mathcal{M}}
\newcommand*{\calN}{\mathcal{N}}
\newcommand*{\calO}{\mathcal{O}}
\newcommand*{\calP}{\mathcal{P}}
\newcommand*{\calQ}{\mathcal{Q}}
\newcommand*{\calR}{\mathcal{R}}
\newcommand*{\calS}{\mathcal{S}}
\newcommand*{\calT}{\mathcal{T}}
\newcommand*{\calU}{\mathcal{U}}
\newcommand*{\calV}{\mathcal{V}}
\newcommand*{\calW}{\mathcal{W}}
\newcommand*{\calX}{\mathcal{X}}
\newcommand*{\calY}{\mathcal{Y}}
\newcommand*{\calZ}{\mathcal{Z}}
% Script Letters
\newcommand*{\scrA}{\mathscr{A}}
\newcommand*{\scrB}{\mathscr{B}}
\newcommand*{\scrC}{\mathscr{C}}
\newcommand*{\scrD}{\mathscr{D}}
\newcommand*{\scrE}{\mathscr{E}}
\newcommand*{\scrF}{\mathscr{F}}
\newcommand*{\scrG}{\mathscr{G}}
\newcommand*{\scrH}{\mathscr{H}}
\newcommand*{\scrI}{\mathscr{I}}
\newcommand*{\scrJ}{\mathscr{J}}
\newcommand*{\scrK}{\mathscr{K}}
\newcommand*{\scrL}{\mathscr{L}}
\newcommand*{\scrM}{\mathscr{M}}
\newcommand*{\scrN}{\mathscr{N}}
\newcommand*{\scrO}{\mathscr{O}}
\newcommand*{\scrP}{\mathscr{P}}
\newcommand*{\scrQ}{\mathscr{Q}}
\newcommand*{\scrR}{\mathscr{R}}
\newcommand*{\scrS}{\mathscr{S}}
\newcommand*{\scrT}{\mathscr{T}}
\newcommand*{\scrU}{\mathscr{U}}
\newcommand*{\scrV}{\mathscr{V}}
\newcommand*{\scrW}{\mathscr{W}}
\newcommand*{\scrX}{\mathscr{X}}
\newcommand*{\scrY}{\mathscr{Y}}
\newcommand*{\scrZ}{\mathscr{Z}}

%Section break
\newcommand*{\brk}{
\rule{2in}{.1pt}
}

%General purpose stuff
\DeclareMathOperator{\Aut}{Aut}
\DeclareMathOperator{\ch}{char}
\DeclareMathOperator{\rank}{rank}
\DeclareMathOperator{\End}{End}
\let\Im\relax
\DeclareMathOperator{\Im}{Im}

%Category Theory
\DeclareMathOperator{\Hom}{Hom}
\let\hom\relax
\DeclareMathOperator{\hom}{hom}
\DeclareMathOperator{\id}{id}
\DeclareMathOperator{\coker}{coker}
\DeclareMathOperator{\colim}{colim}
\DeclareMathOperator{\invlim}{\lim_{\leftarrow}}
\DeclareMathOperator{\dirlim}{\lim_{\rightarrow}}

%Commutative Algebra
\DeclareMathOperator{\gldim}{gldim}
\DeclareMathOperator{\projdim}{projdim}
\DeclareMathOperator{\injdim}{injdim}
\DeclareMathOperator{\findim}{findim}
\DeclareMathOperator{\flatdim}{flatdim}

%Common Categories
\newcommand*{\modR}{\mathbf{mod}\text{-}R}
\newcommand*{\Rmod}{R\text{-}\mathbf{mod}}
\DeclareMathOperator{\Vectk}{\mathbf{Vect}_k}
\DeclareMathOperator{\Ch}{\mathbf{Ch}}
\newcommand*{\Ab}{\mathbf{Ab}}
\newcommand*{\Algk}{\mathbf{Alg}_k}
\newcommand*{\Ring}{\mathbf{Ring}}
\newcommand*{\K}{\mathbf{K}}
\newcommand*{\D}{\mathbf{D}}
\newcommand*{\Db}{\mathbf{D}^b}
\newcommand*{\Dpos}{\mathbf{D}^+}
\newcommand*{\Dneg}{\mathbf{D}^-}
\newcommand*{\Dbperf}{\mathbf{D}^b_{perf}}
\newcommand*{\Dsing}{\mathbf{D}_{sing}}

%Homological algebra
\DeclareMathOperator{\cone}{cone}
\DeclareMathOperator{\HH}{HH}
\DeclareMathOperator{\Der}{Der}
\DeclareMathOperator{\Ext}{Ext}
\DeclareMathOperator{\Tor}{Tor}

%Lie algebras
\DeclareMathOperator{\ad}{ad}
\newcommand*{\gl}{\mathfrak{gl}}
\let\sl\relax
\newcommand*{\sl}{\mathfrak{sl}}
\let\sp\relax
\newcommand*{\sp}{\mathfrak{sp}}
\newcommand*{\so}{\mathfrak{so}}

% Hacks and Tweaks
% Enumerate will automatically use letters (e.g. part a,b,c,...)
\setenumerate[0]{label=(\alph*)}
% Always use wide tildes
\let\tilde\relax
\newcommand*{\tilde}[1]{\widetilde{#1}}
%raise that Chi!
\DeclareRobustCommand{\Chi}{{\mathpalette\irchi\relax}}
\newcommand{\irchi}[2]{\raisebox{\depth}{$#1\chi$}} 



%header stuff
\setlength{\headsep}{24pt}  % space between header and text
\pagestyle{fancy}     % set pagestyle for document
\lhead{Brown Representability} % put text in header (left side)
\rhead{Courts} % put text in header (right side)
\cfoot{\itshape p. \thepage}
\setlength{\headheight}{15pt}
\allowdisplaybreaks

% Document-Specific Macros
\newcommand*{\dgVectk}{\mathbf{dgVect}_k}
\newcommand*{\dgmodA}{\mathbf{dgmod}\text{-}A}
\newcommand*{\pTopc}{\mathbf{Top}_\ast^c}
\newcommand*{\Set}{\mathbf{Set}}
\newcommand*{\Func}{\mathbf{Func}}
\newcommand*{\pCW}{\mathbf{Top}^{CW}_\ast}
\newcommand*{\relCW}{\mathbf{Top}^{CW}_{\hookrightarrow}}

\begin{document}
%make the title page
\title{Brown Representability\vspace{-1ex}}
\author{Nico Courts}
\date{Math 509 -- Spring 2019}
\maketitle

\renewcommand{\abstractname}{Introduction}
\begin{abstract}
  In this document we write an exposition presenting the topic of Brown representability and how it has appeared in different contexts -- beginning first with some necessary preliminaries followed by a more topological perspective and culminating in a modern and categorical statement holds in triangulated categories.
\end{abstract}
\tableofcontents

\newpage
The presentation itself will skip most of the first section, as this will be considered to be prerequisite background knowledge.
Instead, we will jump right in with section 2 discussing the topic at hand.

\section{Some Categorical Background}
In this section we begin with some preliminary definitions and theorems that serve as the machinery for Brown representability. 
The results that follow are fairly standard and the interested reader can find more information in \cite{neeman-book} and \cite{riehl16}.
For the sake of time, we will omit a discussion of these facts in our lecture but include them in this document in case someone is interested in them.
\subsection{General Category Theory}
\begin{defn}
  A \textbf{category} $\calC$ is a collection of \textbf{objects} along with, for every pair $(X,Y)$ of objects in $\calC$, a collection $\Hom_\calC(X,Y)$ of \textbf{morphisms} from $X$ to $Y$ such that
  \begin{itemize}
    \item There is a distinguished map, the \textbf{identity map} $\id_X\in\Hom_\calC(X,X)$ for each $X\in\calC$ such that, for every $f\in\Hom_\calC(X,Y)$ and every $g\in\Hom_\calC(Y,X)$,
    \[f=\id_X\circ f\qquad g=g\circ\id_X.\]
    \item For each $X,Y,$ and $Z$ in $\calC$ there is a map $\circ:\Hom_\calC(Y,Z)\times\Hom_\calC(X,Y)\to\Hom_\calC(X,Z)$ (called \textbf{composition}) that is associative. That is, for each $f\in\Hom_\calC(X,Y)$, $g\in\Hom_\calC(Y,Z)$, and $h\in\Hom_\calC(Z,W)$,
    \[(h\circ g)\circ f=h\circ(g\circ f).\]
  \end{itemize}
\end{defn}
\begin{defn}
	The \textbf{opposite category} $\calC^{op}$ of a category $\calC$ is the category whose objects are 
	precisely those of $\calC$ and whose morphisms are as follows: for every $A,B\in\calC$ and $f\in\Hom_\calC(A,B)$,
	there is a map $f^\ast\in\Hom_{\calC^{op}}(B,A)$ such that for all composable $f$ and $g$ in $\calC$,
	\[(f\circ g)^\ast=g^\ast\circ f^\ast.\]
\end{defn}

\subsection{Size Considerations}
``Size'' is often an issue when you are discussing category theory in that one can quickly get into trouble with Russell's paradox.
To ameliorate this to some degree, it is often wise to restrict oneself to the categories which are suitably small (although often 
ideas can be extended with extra care to a more general setting).

Here we define some of the types of smallness that we are interested in. The general idea to keep in mind is that if $X$ is small, then the 
collections associated to $X$ are sets.

\begin{defn}
	A category $\calC$ is called \textbf{locally small} if for each $A,B\in\calC$, $\Hom_{\calC}(A,B)$ is a set.
\end{defn}
\begin{defn}
	A category $\calC$ is called \textbf{small} if it is locally small and, in addition, has a set's worth of objects.
\end{defn}
\begin{defn}
	A \textbf{small (co)limit} (see \cite{riehl16} for definition of (co)limits) is a limit of a diagram that is indexed by a small indexing category.
\end{defn}
\begin{rmk}
	Since (co)products are (co)limits, this also gives us a definition for \textbf{small (co)products}. We can also just say 
	they are (co)products indexed by a set.
\end{rmk}

\subsection{Functors}
\begin{defn}
  A (covariant) \textbf{functor} between two categories $\calC$ and $\calD$ is a map $F:\calC\to\calD$ that maps objects and morphisms in $\calC$ to those in $\calD$, respectively, such that 
  \begin{itemize}
    \item For each $X\in\calC$, we have $F(\id_X)=\id_{F(X)}$
    \item For each $X,Y\in\calC$ and $g\in\Hom_\calC(X,Y)$, $F(g)$ is a morphism in $\Hom_\calC(F(X),F(Y))$.
    \item For all $X,Y,Z$ in $\calC$ and $f\in\Hom_\calC(X,Y)$ and $g\in\Hom_\calC(Y,Z)$,
    \[F(g\circ f)=F(g)\circ F(f)\]
  \end{itemize}
\end{defn}
\begin{rmk}
	Leveraging duality, one can define a \textbf{contravariant functor} $F:\calC\to \calD$ as a covariant functor $F:\calC^{op}\to \calD$.
\end{rmk}
\begin{defn}
	An \textbf{equivalence of categories} $F:\calC\to\calD$ is a functor such that 
	\begin{itemize}
		\item $F$ is both \textbf{full} (injective on $\Hom$s) and \textbf{faithful} (surjective on $\Hom$s); and 
		\item $F$ is \textbf{essentially surjective} -- for each $D\in\calD$, there is a $D'\in\calD$ such that 
		$D\cong D'$ and, for some $C\in\calC$, we have $F(C)=D'$.
	\end{itemize}
	An \textbf{auto-equivalence} for $\calC$ is an equivalence of categories from $\calC$ to itself.
\end{defn}


\subsection{Natural Transformations}
A natural transformation is a ``map between maps'' that demonstrates some kind of nice compatibility between two functors.
Often, when one says a property is ``natural'' what one actually means is that there is a natural transformation involved.

\begin{defn}
	Let $F,G:\calC\to\calD$ be two functors. Then $\eta:F\to G$, defined to be a collection of \textbf{component maps} $\{\eta_\alpha:F(\alpha)\to G(\alpha)\}_{\alpha\in\calC}$, 
	is called a \textbf{natural transformation} if, for every 
	$A,B\in\calC$ and $f\in\Hom_{\calC}(A,B)$, the following square (called a \textbf{naturality square}) commutes:
	\begin{center}
		\begin{tikzcd}
			F(A)\ar[d,"F(f)"]\ar[r,"\eta_A"] & G(A)\ar[d,"G(f)"]\\
			F(B)\ar[r,"\eta_B"] & G(B)
		\end{tikzcd}
	\end{center}
	If $\eta$ has an inverse, then it is called a \textbf{natural isomorphism.}
\end{defn}

\subsection{Triangulated Categories}
The primary context that we are interested is a class of categories called \textit{triangulated categories.} Examples of 
these include $\K(\modR)$, the homotopy category of chain complexes of $R$-modules (modulo chain homotopy) and $\D(\modR)$,
the derived category in which we (Verdier) localize the previous category at quasi-isomorphisms.
\begin{defn}
	Let $\calC$ be a category with auto-equivalence $\Sigma:\calC\to\calC$. A \textbf{triangle} in $\calC$ is a diagram in $\calC$:
	\[X\to Y\to Z\to\Sigma X.\]
\end{defn}
\begin{defn}
	A \textbf{morphism of triangles} is a commutative diagram
	\begin{center}
		\begin{tikzcd}
			X\ar[r,"u"]\ar[d,"f"]& Y\ar[r,"v"]\ar[d,"g"] & Z\ar[r,"w"]\ar[d,"h"] & \Sigma X\ar[d,"\Sigma f"]\\
			X'\ar[r] & Y'\ar[r] & Z'\ar[r] & \Sigma X'
		\end{tikzcd}
	\end{center}
\end{defn}

\begin{lem}
	The collection of all triangles in $\cal C$ with triangle morphisms form a $k$-linear additive category denoted $\Delta(\cal C)$
\end{lem}

\begin{defn}
	The \textbf{rotation} of a triangle $X\xrightarrow{u} Y\to Z\to \Sigma X$ is $Y\to Z\to \Sigma X\xrightarrow{-\Sigma u}\Sigma Y.$
	We use the notation $\mathscr R(\Delta)$ for the rotation of $\Delta$ and $\scrR^{-1}$ for the inverse rotation.
\end{defn}

\begin{defn}
	Let $\cal C$ be a $k$-linear additive category with a \textbf{suspension} or \textbf{shift} functor (auto-equivalence, actually)
	\[\Sigma:\cal C\to\cal C.\]

	Then $({\calC}, \Sigma ,D)$ is \textbf{pre-triangulated} category where $D$ is a full, nonempty subcategory $D\subseteq\Delta(\cal C)$
	with shift functor $\Sigma$ of $D$ and we have the following axioms:
	\begin{itemize}
		\item (\textbf{TR0}) $0\to X\xrightarrow{\id} X\to 0$ is in $D$ for each $X\in \cal C$ and furthermore $D$ is closed under both shifts and triangle isomorphisms.
		\item (\textbf{TR1}) [\textit{Mapping Cone Axiom}] For any $f:X\to Y$ in $\cal C$, there is a triangle
		\[X\xrightarrow{f} Y\to Z\to\Sigma X\]
		\item (\textbf{TR2}) [\textit{Rotation Axiom}] If $F\in D$, then $\mathscr R(F),\mathscr R^{-1}(F)\in D$
		\item (\textbf{TR3}) [\textit{Morphism Axiom}] Given two triangles 
		\begin{center}
			\begin{tikzcd}
				X\ar[r]\ar[d,"f"] & Y\ar[r]\ar[d,"g"] & Z\ar[r]\ar[d,"h",dashed] & \Sigma X\ar[d,"\Sigma f"]\\
				X'\ar[r] & Y'\ar[r] & Z'\ar[r] & \Sigma X'
			\end{tikzcd}
		\end{center}
		with maps $f$ and $g$, there exists an $h:Z\to Z'$ such that the above diagram commutes.
	\end{itemize}
\end{defn}
\begin{rmk}
	The name ``mapping cone axiom'' in (\textbf{TR2}) above comes from the fact that in $\K(\modR)$, or in fact in 
	any additive category with suspension, $Z$ can be constructed as $\cone(f)$, the mapping cone of $f$. 
\end{rmk}
\begin{defn}
	A triple $(\calC,\Sigma, D)$ is a \textbf{triangulated category} if it is pre-triangulated and, in addition,
	satisfies the axiom (\textbf{TR4}) (Verdier/octahedral axiom):

	Suppose we are given three triangles: $X\xrightarrow{u}Y\to Z'\to\Sigma X$, $Y\xrightarrow{v}Z\to X'\to\Sigma Y$
	and $X\xrightarrow{v\circ u} Z\to Y'\to\Sigma X$. Then there is a triangle $Z'\to Y'\to X'\to\Sigma Z'$ such that
	\begin{center}
		\begin{tikzcd}
			X\ar[dr,"u"]\ar[rr,"v\circ u",bend left] && Z\ar[rr,bend left]\ar[dr] && X'\ar[rr,dashed,bend left]\ar[dr] && \Sigma Z'\\
			& Y\ar[ur,"v"]\ar[dr] && Y'\ar[ur,dashed]\ar[dr] && \Sigma Y\ar[ur] &\\
			&& Z'\ar[rr,bend right]\ar[ur,dashed] && \Sigma X\ar[ur] &&
		\end{tikzcd}
	\end{center}
\end{defn}
\begin{rmk}
There are two other representations of the octahedral axiom that are sometimes more helpful:
\begin{center}
	\begin{tikzcd}
		X\ar[r,"u"]\ar[d,"\sim"] & Y\ar[r]\ar[d,"v"] & Z'\ar[r]\ar[d,dashed] &\Sigma X\ar[d,"\sim"]\\
		X\ar[r,"v\circ u"] & Z\ar[r]\ar[d] & Y'\ar[r]\ar[d,dashed] &\Sigma X\ar[d,"\Sigma u"]\\
		& X'\ar[r,"\sim"] \ar[d] & X'\ar[r]\ar[d,dashed] & \Sigma Y\\
		&\Sigma Y\ar[r] & \Sigma Z' &
	\end{tikzcd}
\end{center}

\begin{center}
	\begin{tikzcd}
		&& & X\ar[dl,"u"]\ar[dd,"v\circ u"] &\\
		&& Y\ar[dr,"v"]\ar[ddll] & &\\
		&& & Z\ar[d]\ar[dr] &\\
		Z'\ar[rrr,dashed] && & Y'\ar[r,dashed] & X'
	\end{tikzcd}
\end{center}
\end{rmk}

We often want that our triangulated category $\calT$ is slightly more controlled. For this, we introduce another 
condition:
\begin{defn}
  If $\calT$ is a triangulated category that is closed under small coproducts, then it is said to satisfy (\textbf{TR5}). If $\calT^{op}$ satisfies (\textbf{TR5}) (or equivalently $\calT$ is closed under small products), then we say $\calT$ satisfies (\textbf{TR5}*).
\end{defn}

\section{Brown Representability in Traditional and Stable Homotopy Theory}
One of the first incarnations of Brown representability comes from topology (as these things often do). First we need to single out 
the appropriate context.

\subsection{Category of Interest}
\begin{defn}
	Let $\pTopc$ denote the category of connected pointed topological spaces. That is, the objects are pairs $(X,p)$ where $X$ is a connected 
	topological space and $p$ is a distinguished point in $X$. The maps are continuous maps that send the distinguished point of one pair to the distinguished point in the other.
\end{defn}
\begin{defn}
	$\text{Ho}(\pTopc)$ is the \textbf{homotopy category of connected topological spaces.} The objects are the same as in $\pTopc$, 
	but maps are \textit{weak homotopy equivalence classes} (relative to the basepoint) of maps of pointed spaces.
\end{defn}
\begin{rmk}
	In essence, $\text{Ho}(\pTopc)$ consists of weak homotopy equivalence classes (recall that two objects are weak homotopy equivalent if they have the same 
	homotopy groups).
\end{rmk}
\begin{rmk}
	This is a form of \textit{localization of categories} that is outside the scope of this paper, but closely mirrors the idea 
	we see in commutative algebra.
\end{rmk}
\begin{rmk}
	Since any topological space is weakly homotopy equivalent to a CW-complex (see next remark), we can easily see that 
	$\text{Ho}(\pTopc)\simeq\text{Ho}(\mathbf{\pCW})$, the homotopy category of pointed CW complexes. This gives us some nice leverage for computation and tangibility.
\end{rmk}
\begin{rmk}
	There is an iterative, constructive argument that leads to this (see e.g. \cite{hatcher}), but it ends up that 
	a cleaner way to see this is by considering the Dold-Kan correspondence between simplicial 
	abelian groups and simplicial complexes of connective cochains of abelian groups. Connectivity means that the homology is concentrated in positive degree. 
	Then constructing the so-called Eilenberg-Maclane spaces $K(G,n)$ amount to writing down a cochain complex and applying a functor.
\end{rmk}



\subsection{Ancillary Definitions}
\begin{defn}\label{def-weak-pushouts}
	A \textbf{weak pushout} in $\pCW$ of a diagram $X\xleftarrow{f}Y\xrightarrow{g}Z$ is an object $W$ and maps 
	$f':X\to W$ and $g':Z\to W$ such that for any object $O$ and maps $\alpha:X\to O$ and $\beta:Z\to O$, 
	such that $\alpha\circ f=\beta\circ g$, there exists a map $\gamma:W\to O$ such that the following diagram commutes:
	\begin{center}
		\begin{tikzcd}
			Y\ar[r,"f"]\ar[d,"g"] & X\ar[d,"f'"]\ar[ddr,"\alpha",bend left] &\\
			Z\ar[r,"g'"]\ar[rrd,"\beta",bend right] & W\ar[dr,dashed,"\gamma"] & \\
			& & O 
		\end{tikzcd}
	\end{center}

	A \textbf{weak pullback} is defined dually.
\end{defn}
\begin{rmk}
	The ``weakness'' referred to in this definition is due to the fact that the map $\gamma$ exists, but is not 
	required to be unique (unlike the regular pullback and pushout).
\end{rmk}

\begin{defn}\label{def-rep-func}
	Let $F:\calC\to \mathbf{Set}$ be a (covariant) functor. Then $F$ is called \textbf{representable} if there exists an object $A\in \calC$ such that
	$F$ is naturally isomorphic to $\Hom_{\calC}(-,A)$.
\end{defn}

\subsection{The Theorem}
The following result was proven by Brown and presented in \cite{brown65} and further generalized in \cite{adams71}.
\begin{thm}[Brown `65]\label{thm-brown}
	A contravariant functor $F:\text{Ho}(\pTopc)\simeq\text{Ho}(\pCW)\to\mathbf{Set}_\ast$ is representable precisely if 
	\begin{itemize}
		\item $F$ takes coproducts to products; and 
		\item $F$ takes weak pushouts to weak pullbacks.
	\end{itemize}
\end{thm}
\begin{defn}
	A \textbf{Brown functor} is a functor $F:\text{Ho}(\pCW)^{op}\to\Set$ that takes coproducts to products and weak pushouts to weak pullbacks.
\end{defn}
\begin{rmk}
	We will delay the proof of this theorem to the end, where we prove it in the more general setting. It is a simple application 
	once one realizes that $\text{Ho}(\pCW)$ is a triangulated category and that a Brown functor (i.e. one that 
	satisfies the hypotheses above) is a cohomological functor (def~\ref{def-hom-func}).
\end{rmk}

\subsection{Generalized (Eilenberg-Steenrod) Cohomology Theories}
The idea here is that we can construct a general definition that generally captures what we mean when we say we are 
working with cohomology.

Notice that these theories notably do \textit{not} include nonabelian cohomology theories and other theories of interest.
\begin{defn}
	$\Ab^\bbZ$ is the functor category $\Func(\bbZ,\Ab)$ or, equivalently, the category of $\bbZ$-graded Abelian groups.
\end{defn}
\begin{rmk}
	In case you haven't seen it before, the category $\bbZ$ we're talking about has as its objects the integers and 
	a (unique) morphism $m\to n$ whenever $n\ge m$. This is the category we can construct from any set that admits a preorder (reflexivity and transitivity).

	Functorality here says that for any $k\in\bbN$ we can consider the degree $k$ maps between the $i^{th}$ graded piece 
	of $G$ and the $(i+k)^{th}$ graded piece of $H$ for any $G,H\in\Ab^{\bbZ}$. If $\varphi:G\to H$ is any degree zero map and $h\in H^k$ is 
	any degree $k$ element in $H$, then an example of such a map is $\varphi'(g)=h\cdot \varphi(g).$
\end{rmk}

\begin{defn} 
	$\relCW$ is the category of pairs $(X,A)$ of CW complexes with inclusions $A\hookrightarrow X$ with subspace-preserving 
morphisms.
\end{defn}

\begin{defn}[(Unreduced, Relative) Generalized Cohomology Theory]
	An unreduced, relative generalized cohomology theory for $\relCW$ is a collection of functors 
	\[E^\bullet:\text{Ho}(\relCW)\to\Ab^\bbZ\]
	along with natural isomorphisms
	\[\eta_{(X,A)}:E^\bullet(A,\varnothing)\to E^{\bullet+1}(X,A)\]
	such that it satisfies the following: 
	\begin{itemize}
		\item (\textit{Homotopy Invariance}) If $f,g\in\Hom((X,A),(Y,B))$ and $f\sim_Ag$ (that is, $f$ is homotopic to )
		\item (\textit{Exactness}) For $(X,A)$, the following is exact:
		\[\cdots\to E^n(X,A)\to E^n(X,\varnothing)\to E^n(A,\varnothing)\xrightarrow{\eta_{(X,A)}}E^{n+1}(X,A)\to\cdots\]
		or, more succinctly,
		\[\cdots\to E^n(X,A)\to E^n(X)\to E^n(A)\to E^{n+1}(X,A)\]
		which reflects the long exact sequence in relative cohomology that you may be familiar with (see your favorite book on algebraic topology, e.g.
		\cite[p.199]{hatcher}).
		\item (\textit{Excision}) For $(X,A)$ and $(A,U)$ in $\relCW$ such that $\overline{U}\subseteq \text{Int}(A),$ the
		inclusion $\iota:(X\setminus U,A\setminus U)\hookrightarrow (X,A)$ induces an isomorphism
		\[\iota^\ast:E^\bullet(X,A)\xrightarrow{\simeq}E^\bullet(X\setminus U,A\setminus U)\]
	\end{itemize}
\end{defn}
\begin{prop}
	Generalized cohomology theories are Brown functors.
\end{prop}
\begin{rmk}
	Notice that technically the definition of a Brown functor is a map from $\pCW$, which embeds as a proper subcategory osf $\relCW$. However 
	the homotopy category is still going to be an $(\infty,1)$-category, so the more general version of Brown representability applies.
\end{rmk}

\subsection{Stable Homotopy Theory}
The point of this next section is to generalize further to allow us to place this in a broader context in which 
computing the cohomology of topological spaces is just a special case.

Many of the definitions and results here are due to \cite{lurie}. It isn't particularly illuminating (and perhaps out of scope for this paper)
to look at formal definitions in some cases, so I will rely on intuition.

\begin{rmk}
	A 2-category is a weakening of a category in the following way: a 2-category $\calC$ is a category that is weakly enriched over $\mathbf{Cat}$.
	That is, the $\Hom$ objects are themselves categories and the usual identity and associativity axioms are replaced 
	with weaker identity and associativity ``up to 2-isomorphism.''

	Then we can define iteratively an $n$-category as a category that is weakly enriched over an $(n-1)$-category. Then eschewing formalities,
	we can think of an $\infty$-category as the limit of this iterative process.
\end{rmk}

\begin{defn}
	An $(\infty,1)$-category is an $\infty$-category in which all $k$-morphisms for $k\ge 2$ are invertible.
\end{defn}
\begin{prop}
	$\text{Ho}(\pCW)$ is an $(\infty,1)$-category.
\end{prop}
\begin{rmk}
	The idea here is that $(\infty,1)$-categories are the proper context to do work involving (higher) homotopies. They 
	capture the notion of maps between maps. This introduces the notion of two maps being ``not identical, but homotopic''. 
	
	We want to say that (e.g.) if the composition $f\circ g$ of two loops is 
	contractible, that the composition is essentially (up to homotopy) the constant map. The standard category of topological spaces 
	and continuous maps simply doesn't allow for this. 
\end{rmk}

\begin{defn}
	Let $\calC$ be an $(\infty,1)$-category with homotopy (or weak) pullbacks (see defn~\ref{def-weak-pushouts}). Then the \textbf{loop space object}
	for $X\in\calC$ is the homotopy pullback of the diagram $\ast\to X\leftarrow\ast$:
	\begin{center}
		\begin{tikzcd}
			\Omega X\ar[r]\ar[d] & \ast\ar[d]\\
			\ast\ar[r] & X
		\end{tikzcd}
	\end{center}

	Dually, if $\calC$ has homotopy pushouts, we can define a \textbf{suspension object} for $X$, $\Sigma X$, to be the homotopy pushout of 
	$\ast\leftarrow X\to \ast.$
\end{defn}

Finally we get the thing we're looking for: the context for performing stable homotopy theory.
\begin{defn}
	A \textbf{stable $(\infty,1)$-category} is an $(\infty,1)$-category that is stable under the looping and suspension operations. In particular, 
	we get that $\Omega:\calC\to\calC$ is an equivalence with inverse $\Sigma.$
\end{defn}
\begin{defn}
	A \textbf{sequential spectrum} (in a category where we have suspensions) is a sequence 
	\[X_0\to X_1\to\cdots\]
	along with structural morphisms $\sigma:\Sigma X_n\to X_{n+1}$.
\end{defn}
\begin{rmk}
	In general one can talk about spectrum objects in an $(\infty,1)$-category, but this is more tangible and works for $\pCW$.
\end{rmk}

\subsection{What This Means}
What should we take away from Brown representability? If we accept the hypothesis that generalized cohomology 
theories are reasonable approximations of what we are looking for in cohomology, this implies that each such 
incarnation of a cohomology theory $\tilde E^\bullet$ admits a spectrum object $E_\bullet$ in the stable homotopy category 
such that (degree-wise) $\tilde E^\bullet\cong \text{Ho}(\pCW)(-,E_\bullet)$ such that the suspension isomorphism 
$\eta:\tilde E^n(-)\to \tilde E^{n+1}(\Sigma(-))$ is induced from the structure morphisms $\sigma_n:E_n\to\Omega E_{n+1}$ of the spectrum:
\begin{align*}
	\tilde E^n(-)\cong \text{Ho}(\pCW)(-,E_n)\xrightarrow{(\sigma_n)}&\text{Ho}(\pCW)(-,\Omega E_{n+1})\\
	&\cong\text{Ho}(\pCW)(\Sigma(-),E_{n+1})\cong \tilde E^{n+1}(\Sigma(-))
\end{align*}
Because of this, we tend to think of 
\[\tilde E^\bullet(X)=H(X,E^\bullet)\]
where $E^\bullet$ is the spectrum object representing $\tilde E^\bullet$. In other words, \textit{every generalized cohomology theory is just 
cohomology with coefficients in some spectrum object.}
\begin{rmk}
	Notice that there is actually something a little funny here, since $E^\bullet$ is a spectrum object, and not an element of $\text{Ho}(\pCW)$ proper.
	So saying that $E^\bullet$ is a representing object in the traditional sense is perhaps misguided although clearly the fact that we get a 
	spectrum with structure morphisms inducing the suspension isomorphisms is nothing to scoff at.
\end{rmk}

\subsection{For (ordinary) cohomology}
If we restrict our attention further to ordinary cohomology theories, that is, $E^\bullet$ such that $E^i(S^0)=0$ whenever $i\ne 0$,
we get an exceptionally nice result. Generally we write $E^\bullet=H^\bullet(-,A)$ where $A$ is an abelian group. 
The representing spectrum object for $E^\bullet$ is the so-called \textbf{Eilenberg-Maclane spectrum} $HA$:
\[HA_n=K(A,n).\]

\section{Brown Representability for Triangulated Categories}
\subsection{Additional Notions}
Along with the standard categorical definitions we saw in section one, we need one further restriction on our categories as formulated in \cite{neeman-article}:
\begin{defn}[Compact Object]
	Let $X\in\calC$ be an object in a triangulated category. Then $X$ is called \textbf{compact} if for every coproduct of objects in $\calC$
	\[\Hom_\calC\left(X,\coprod_{\lambda\in\Lambda}t_\lambda\right)=\coprod_{\lambda\in\Lambda}\Hom_\calC(X,t_\lambda)\]
\end{defn}
\begin{rmk}
	The idea here is that when the codomain is a product, this equality always exists. The projections give you 
	a product of maps into the factors and the universal property of products gives you a (unique!) map into the product if you
	have maps into each of the factors.

	This does not hold in general for coproducts, however! Recall that in additive categories 
	one has a canonical isomorphism between finite products and coproducts. Since triangulated 
	categories are additive, this is akin to saying that $X$ is well-behaved in that the maps 
	from $X$ still split even when the codomain is an arbitrarily large coproduct.
\end{rmk}
\begin{defn}[Compactly Generated Triangulated Category]
	Let $\calC$ be a triangulated category. Then $\calC$ is called \textbf{compactly generated by a set $G$} of its elements 
	if $\calC$ is closed under small coproducts and $G$ consists of compact objects of $\calC$ such that for all $X\in\calC$,
	\[\Hom_\calC(G,X)=0\quad\Rightarrow\quad X=0.\]
\end{defn}
\begin{rmk}
	Here the generators of $\calC$ are elements that have ``sufficient complexity'' to detect all nonzero elements. 
	
	In the more general context where $\calC$ is not necessarily additive, we can generalize this definition to a set of objects 
	such that for every generator $E\in G$ and any $X,Y\in\calC$, if for every $f,g\in\Hom(X,Y)$ and $h\in\Hom(E,X)$ we have 
	\[f\circ e=g\circ e\]
	then 
	\[f=g\]
	and here we would say that $G$ has ``sufficient complexity'' to differentiate between any two non-identical maps.
\end{rmk}
\begin{ex}
	In $\Rmod$, a set of generators is the collection of all free $R$-modules or, equivalently, the collection 
	\[\{R^S|S\in\Set\setminus\varnothing\}\]
	where if $S\subset M\in\Rmod$ is any nonempty subset, then $R^S$ is the product of $|S|$ copies of $R$, 
	where a nonzero map to $M$ is the map which sends the identity in factor to a different $s\in S$.
\end{ex}
\begin{defn}\label{def-hom-func}
	Let $R$ be a ring and let $\calT$ be a triangulated category. Then a functor $F:\calT\to\Rmod$ is called \textbf{homological}
	if $F$ sends each triangle $A\to B\to C\to\Sigma A$ to an exact sequence in $\Rmod$.
\end{defn}
\begin{rmk}
	This definition extends to functors from triangulated categories to any Abelian category, but due to the Freyd-Mitchell
	embedding it suffices to use $\Rmod$ for all small examples.
\end{rmk}
\begin{defn}
	In a triangulated category with (small) coproducts, the \textbf{homotopy colimit} of a sequence $X_0\to X_1\to\cdots$, denoted $\operatorname{hocolim}_{\rightarrow}X_i$,
	is the object fitting into the triangle 
	\[\coprod X_i\xrightarrow{\text{shift}}\coprod X_i\to \operatorname{hocolim}_\rightarrow X_i \to\Sigma\prod X_i\]
	where the shift map above is the one following the maps in the given sequence.
\end{defn}


\subsection{The Statement}
Finally we come to the full-powered version of the Brown Representability due to Neeman in \cite{neeman-article}:
\begin{thm}[Brown-Neeman]
	Let $\calT$ be a compactly generated triangulated category. For a functor $H:\calT^{op}\to\Ab$, the following 
	are equivalent:
	\begin{itemize}
		\item The functor $H$ is homological (def~\ref{def-hom-func}) and preserves small coproducts.
		\item $H$ is representable (def~\ref{def-rep-func}).
	\end{itemize}
\end{thm}
\begin{prf}
	Begin by letting $T$ be a set of compact generators for $\calT.$ Then define 
	\[U_0=\bigcup_{t\in T}H(t)=\{(\alpha,t)\}\]
	where the right equality is a matter of notation (remembering the $t$ from which $\alpha$ came).

	Then set $X_0=\coprod_{(\alpha,t)\in U_0}t$, so (by pulling out coproducts):
	\[H(X_0)=\prod_{(\alpha,t)\in U_0}H(t).\]
	
	Let $\alpha_0\in H(X_0)$ be the element which has $\alpha$ in the coordinate corresponding to the $(\alpha,t)$ spot
	(for all $(\alpha,t)\in U_0$). Notice that if $\iota:t\hookrightarrow X_0$ is inclusion of $t$ into the position corresponding to $(\alpha,t)$,
	then $H(\iota)(\alpha_0)=\alpha.$

	Now by Yoneda lemma, such an element in $H(X_0)$ corresponds to a natural transformation $\eta_0:\Hom(-,X_0)\to H$. The above argument 
	shows that the component morphisms $\eta^0(t):\Hom(t,X_0)\to H(t)$ are surjective maps.

	\brk

	For the induction step, assume that we have $X_i\in\calT$ and $\eta^i:\Hom(-,X_i)\to H$ for $i\ge 0$.
	Define 
	\[U_{i+1}=\bigcup_{t\in T}\ker\eta^i_t\]
	where each pair $(f,t)$ has the property that $\eta^i_t(f)=0.$

	Let $K_{i+1}=\coprod_{(f,t)}t$ and $F:K_{i+1}\to X_i$ be the map such that the restriction to the $(f,t)^{th}$
	summand in $K_{i+1}$ is $f$. Using the mapping cone axiom, complete $F$ to a triangle:
	\[K_{i+1}\xrightarrow{F}X_i\to X_{i+1}\to\Sigma K_{i+1}\]
	and define $X_{i+1}$ to be the element fitting into this extension.

	Applying Yoneda, extract $\alpha_i$ from $\eta^i$. Then consider that $H(F):H(X_i)\to H(K_{i+1})$ is the product of maps 
	with $H(f)$ in the $(f,t)^{th}$ spot. But (by the details of the Yoneda equivalence),
	\[\eta^i(f)=H(f)(\alpha_i)\]
	for all $f\in Hom(t,X_i)$. But then since the $f$ we chose in our indexing set $U_{i+1}$ were precisely 
	those such that $\eta^i(f)=0$, we can conclude that $H(f)(\alpha_i)=0$, and so $H(F)(\alpha_i)=0$.

	Now using that $H$ is a cohomological functor, we get an exact sequence 
	\[H(X_{i+1})\xrightarrow{k} H(X_i)\xrightarrow{H(F)} H(K_{i+1})\]
	and since $\alpha_i$ goes to zero on the right, there is an $\alpha_{i+1}\in H(X_{i+1})$ such that 
	it is mapped to $\alpha_i$ under $k$.

	Again leveraging Yoneda, let $\alpha_{i+1}$ correspond to $\eta^{i+1}$. This gives us a commutative diagram 
	\begin{center}
		\begin{tikzcd}
			\Hom(-,X_i)\ar[dr,"\eta^i"]\ar[d,"k",swap] & \\
			\Hom(-,X_{i+1})\ar[r,swap,"\eta^{i+1}"] & H
		\end{tikzcd}
	\end{center}
	This completes the induction step, giving us our $X_{i+1}$ and $\eta^{i+1}$.

	\brk

	From here the statement follows once we realize these two facts: If we set $X=\operatorname{hocolim}_{\to}X_i$,
	\begin{itemize}
		\item There is a natural transformation $\xi:\Hom(-,X)\to H$ such that $\eta^i$ factors through $\xi$.
		\item $\xi$ is a natural isomorphism.
	\end{itemize}
\end{prf}
\begin{rmk}
	The rest of the details here are mostly technical and not particularly enlightening, since we have the gist 
	of what we wanted: $X$, the homotopy colimit of these objects we constructed. This is our representing object.
\end{rmk}

\subsection{Brown Representability and Adjointness}
\begin{defn}
	A \textbf{triangulated functor} between two triangulated categories $\calT$ and $\calS$ is an additive functor that 
	commutes with rotation and takes distinguished triangles to distinguished triangles.
\end{defn}
The last (and sometimes more useful) statement to come from Brown representability involves adjointness:
\begin{cor}
	Let $F:\calT\to\calS$ be a triangulated functor from $\calT$, a triangulated category, to $\calS$, a compactly-generated triangulated category, such that $F$ respects coproducts. 
	Then $F$ has a right adjoint $G$.
\end{cor}
\begin{rmk}
	The idea for the proof here is to take a set $T$ of compact generators for $\calS$ and for each to construct $\Phi_t:\calS\to\Set$:
	\[\Phi_t(s)=\Hom_\calT(F(s),t).\]
	One can show that $\Phi_t$ is a cohomological functor for all $t$ and further that it takes coproducts to products, so Brown applies.
	Thus there is an object $G(t)$ for each $t$ such that 
	\[\Hom_\calT(F(s),t)\cong\Hom_\calS(s,G(t))\]
	and it can be shown that this extends to a functor (which is then clearly adjoint to $F$).
\end{rmk}

\subsection{Application}
The primary application that Neeman cites in his paper is that of Grothendieck duality:
if $f:X\to Y$ is a proper morphism of Noetherian schemes of finite dimension, the right derived
functor $Rf_\ast$ of its direct image functor has a right adjoint -- generally denoted $f^!$

In other words, there is a natural isomorphism 
\[Rf_\ast \calR\Hom_X(x,f^!y)\cong\calR\Hom_Y(Rf_\ast x,y)\]
where $x$ is a quasi-coherent sheaf over $X$ and $y$ is a quasi-coherent sheaf over $Y$. Notice this isomorphism 
is in the bounded-below derived category of quasicoherent sheaves over $Y$: $\Dpos(qc/Y)$.

If you can prove that $\Dpos(qc/X)$ is triangulated and that $Rf_\ast$ is suitably nice, this gives you 
the adjoint without having to do the explicit construction carried out by Grothendieck when he originally 
established the existence of $f^!$.

%%%%%%%%%%%%%%%%%%%%%%%%%%%%%%%%%%%%%%
%%%%%%%%%%  Bibliography %%%%%%%%%%%%%
%%%%%%%%%%%%%%%%%%%%%%%%%%%%%%%%%%%%%%
\medskip

\printbibliography

\end{document}
